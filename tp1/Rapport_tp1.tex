\documentclass[]{article}
\usepackage{lmodern}
\usepackage{amssymb,amsmath}
\usepackage{ifxetex,ifluatex}
\usepackage{fixltx2e} % provides \textsubscript
\ifnum 0\ifxetex 1\fi\ifluatex 1\fi=0 % if pdftex
  \usepackage[T1]{fontenc}
  \usepackage[utf8]{inputenc}
\else % if luatex or xelatex
  \ifxetex
    \usepackage{mathspec}
  \else
    \usepackage{fontspec}
  \fi
  \defaultfontfeatures{Ligatures=TeX,Scale=MatchLowercase}
\fi
% use upquote if available, for straight quotes in verbatim environments
\IfFileExists{upquote.sty}{\usepackage{upquote}}{}
% use microtype if available
\IfFileExists{microtype.sty}{%
\usepackage{microtype}
\UseMicrotypeSet[protrusion]{basicmath} % disable protrusion for tt fonts
}{}
\usepackage[margin=1in]{geometry}
\usepackage{hyperref}
\hypersetup{unicode=true,
            pdfborder={0 0 0},
            breaklinks=true}
\urlstyle{same}  % don't use monospace font for urls
\usepackage{color}
\usepackage{fancyvrb}
\newcommand{\VerbBar}{|}
\newcommand{\VERB}{\Verb[commandchars=\\\{\}]}
\DefineVerbatimEnvironment{Highlighting}{Verbatim}{commandchars=\\\{\}}
% Add ',fontsize=\small' for more characters per line
\usepackage{framed}
\definecolor{shadecolor}{RGB}{248,248,248}
\newenvironment{Shaded}{\begin{snugshade}}{\end{snugshade}}
\newcommand{\KeywordTok}[1]{\textcolor[rgb]{0.13,0.29,0.53}{\textbf{#1}}}
\newcommand{\DataTypeTok}[1]{\textcolor[rgb]{0.13,0.29,0.53}{#1}}
\newcommand{\DecValTok}[1]{\textcolor[rgb]{0.00,0.00,0.81}{#1}}
\newcommand{\BaseNTok}[1]{\textcolor[rgb]{0.00,0.00,0.81}{#1}}
\newcommand{\FloatTok}[1]{\textcolor[rgb]{0.00,0.00,0.81}{#1}}
\newcommand{\ConstantTok}[1]{\textcolor[rgb]{0.00,0.00,0.00}{#1}}
\newcommand{\CharTok}[1]{\textcolor[rgb]{0.31,0.60,0.02}{#1}}
\newcommand{\SpecialCharTok}[1]{\textcolor[rgb]{0.00,0.00,0.00}{#1}}
\newcommand{\StringTok}[1]{\textcolor[rgb]{0.31,0.60,0.02}{#1}}
\newcommand{\VerbatimStringTok}[1]{\textcolor[rgb]{0.31,0.60,0.02}{#1}}
\newcommand{\SpecialStringTok}[1]{\textcolor[rgb]{0.31,0.60,0.02}{#1}}
\newcommand{\ImportTok}[1]{#1}
\newcommand{\CommentTok}[1]{\textcolor[rgb]{0.56,0.35,0.01}{\textit{#1}}}
\newcommand{\DocumentationTok}[1]{\textcolor[rgb]{0.56,0.35,0.01}{\textbf{\textit{#1}}}}
\newcommand{\AnnotationTok}[1]{\textcolor[rgb]{0.56,0.35,0.01}{\textbf{\textit{#1}}}}
\newcommand{\CommentVarTok}[1]{\textcolor[rgb]{0.56,0.35,0.01}{\textbf{\textit{#1}}}}
\newcommand{\OtherTok}[1]{\textcolor[rgb]{0.56,0.35,0.01}{#1}}
\newcommand{\FunctionTok}[1]{\textcolor[rgb]{0.00,0.00,0.00}{#1}}
\newcommand{\VariableTok}[1]{\textcolor[rgb]{0.00,0.00,0.00}{#1}}
\newcommand{\ControlFlowTok}[1]{\textcolor[rgb]{0.13,0.29,0.53}{\textbf{#1}}}
\newcommand{\OperatorTok}[1]{\textcolor[rgb]{0.81,0.36,0.00}{\textbf{#1}}}
\newcommand{\BuiltInTok}[1]{#1}
\newcommand{\ExtensionTok}[1]{#1}
\newcommand{\PreprocessorTok}[1]{\textcolor[rgb]{0.56,0.35,0.01}{\textit{#1}}}
\newcommand{\AttributeTok}[1]{\textcolor[rgb]{0.77,0.63,0.00}{#1}}
\newcommand{\RegionMarkerTok}[1]{#1}
\newcommand{\InformationTok}[1]{\textcolor[rgb]{0.56,0.35,0.01}{\textbf{\textit{#1}}}}
\newcommand{\WarningTok}[1]{\textcolor[rgb]{0.56,0.35,0.01}{\textbf{\textit{#1}}}}
\newcommand{\AlertTok}[1]{\textcolor[rgb]{0.94,0.16,0.16}{#1}}
\newcommand{\ErrorTok}[1]{\textcolor[rgb]{0.64,0.00,0.00}{\textbf{#1}}}
\newcommand{\NormalTok}[1]{#1}
\usepackage{graphicx,grffile}
\makeatletter
\def\maxwidth{\ifdim\Gin@nat@width>\linewidth\linewidth\else\Gin@nat@width\fi}
\def\maxheight{\ifdim\Gin@nat@height>\textheight\textheight\else\Gin@nat@height\fi}
\makeatother
% Scale images if necessary, so that they will not overflow the page
% margins by default, and it is still possible to overwrite the defaults
% using explicit options in \includegraphics[width, height, ...]{}
\setkeys{Gin}{width=\maxwidth,height=\maxheight,keepaspectratio}
\IfFileExists{parskip.sty}{%
\usepackage{parskip}
}{% else
\setlength{\parindent}{0pt}
\setlength{\parskip}{6pt plus 2pt minus 1pt}
}
\setlength{\emergencystretch}{3em}  % prevent overfull lines
\providecommand{\tightlist}{%
  \setlength{\itemsep}{0pt}\setlength{\parskip}{0pt}}
\setcounter{secnumdepth}{0}
% Redefines (sub)paragraphs to behave more like sections
\ifx\paragraph\undefined\else
\let\oldparagraph\paragraph
\renewcommand{\paragraph}[1]{\oldparagraph{#1}\mbox{}}
\fi
\ifx\subparagraph\undefined\else
\let\oldsubparagraph\subparagraph
\renewcommand{\subparagraph}[1]{\oldsubparagraph{#1}\mbox{}}
\fi

%%% Use protect on footnotes to avoid problems with footnotes in titles
\let\rmarkdownfootnote\footnote%
\def\footnote{\protect\rmarkdownfootnote}

%%% Change title format to be more compact
\usepackage{titling}

% Create subtitle command for use in maketitle
\newcommand{\subtitle}[1]{
  \posttitle{
    \begin{center}\large#1\end{center}
    }
}

\setlength{\droptitle}{-2em}
  \title{}
  \pretitle{\vspace{\droptitle}}
  \posttitle{}
  \author{}
  \preauthor{}\postauthor{}
  \date{}
  \predate{}\postdate{}


\begin{document}

\newpage

\begin{flushright}
    Équipe no 4 
\end{flushright}

\begin{center}
    \vspace{5\baselineskip}
    Nicholas Langevin \\
    (111 184 631) \\
    \vspace{2\baselineskip}
    Alexandre Turcotte \\
    (111 172 613) \\
    \vspace{8\baselineskip}
    Mathématiques actuarielles IARD 1 \\
    ACT-2005 \\
    \vspace{8\baselineskip}
    \textbf{Travail pratique 1} \\
    \vspace{8\baselineskip}
    Travail présenté à \\
    Andrew Luong \\
    \vspace{8\baselineskip}
    École d’actuariat \\
    Université Laval \\
    Automne 2018
\end{center}

\section{Question 1}\label{question-1}

\subsubsection{a) Estimation du coefficient
d'asymétrie}\label{a-estimation-du-coefficient-dasymetrie}

\includegraphics{Rapport_tp1_files/figure-latex/unnamed-chunk-1-1.pdf}

À partir de l'histogramme, il est possible de noter que la distribution
est plus dense et concentrée à gauche et que la queue de la distribution
tend vers la droite. Par conséquent, la distribution n'est pas
symétrique et le coefficient d'asymétrie devrait être positif. Cela
concorde effectivement avec le coefficient d'asymétrie estimé
empiriquement puisqu'il est de 1.3076743 conparativement à celui de la
loi normale qui est égal à 0. La loi normale a une distribution
symétrique et comme le coefficient d'asymétrie des valeurs simulées est
plus grand que celui de la loi normale, cela implique que la
distribution est asymétrique vers la droite. Donc, elle possède une
queue de distribution à droite comme il est possible d'observer sur
l'histogramme précédent.

\newpage

\subsubsection{b) Intervalle de confiance pour le coefficient
d'asymétrie}\label{b-intervalle-de-confiance-pour-le-coefficient-dasymetrie}

Avec la méthode de ré-échantillonnage, la variance estimée pour le
coefficient d'asymétrie est de 0.3142622. À partir de l'estimateur
ponctuel du coefficient d'asymétrie calculé en a) et de cette variance
estimée, il est possible d'obtenir l'intervalle de confiance suivant
pour le coefficient d'asymétrie :

\begin{align*}
[0.2089363, 2.4064123]
\end{align*}

\subsubsection{c) Coefficient d'asymétrie
théorique}\label{c-coefficient-dasymetrie-theorique}

Les moments de la loi exponentielle sont donnés par:

\begin{minipage}{5cm}
    \begin{align*}
        E[x]   &= \left. M_x^{'}(t) \right|_{t=0} \\
       &= \frac{d}{dt} \left.\left( \frac{\theta}{\theta - t} \right)\right|_{t=0} \\
       &= \left.\left( \frac{\theta}{(\theta - t)^2} \right)\right|_{t=0} \\
       &= \frac{1}{\theta}
    \end{align*}
\end{minipage}\begin{minipage}{5cm}
    \begin{align*}
        E[x^2] &= \left. M_x^{''}(t) \right|_{t=0} \\
       &= \frac{d}{dt} \left.\left( \frac{\theta}{(\theta - t)^2} \right)\right|_{t=0} \\
       &= \left.\left( \frac{2 \theta}{(\theta - t)^3} \right)\right|_{t=0} \\
       &= \frac{2}{\theta^2}
    \end{align*}
\end{minipage}\begin{minipage}{5cm}
    \begin{align*}
        E[x^3] &= \left. M_x^{'''}(t) \right|_{t=0} \\
       &= \frac{d}{dt} \left.\left( \frac{2 \theta}{(\theta - t)^3} \right)\right|_{t=0} \\
       &= \left.\left( \frac{6 \theta}{(\theta - t)^4} \right)\right|_{t=0} \\
       &= \frac{6}{\theta^3}
    \end{align*}
\end{minipage}

Ainsi, le coeficient d'asymétrie théorique est donnée par

\begin{align*}
    \gamma 
    &= \frac{E\left[ (x - \mu)^3 \right]}{\sigma^3} \\
    &= \frac{1}{\sigma^3} \left( E[x^3 - 3x^2\mu + 3x\mu^2 - \mu^3] \right) \\
    &= \frac{1}{\sigma^3} \left( E[x^3] - 3\mu E[x^2] + 3\mu^2E[x] - \mu^3  \right) \\
    &= \theta^3 \left[ \frac{6}{\theta^3} - 3 \left(\frac{1}{\theta}\right) \left(\frac{2}{\theta^2}\right) + 3 \left(\frac{1}{\theta}\right)^3 - \left(\frac{1}{\theta}\right)^3 \right] \\
    &= \theta^3 \left[ \frac{6}{\theta^3} - \frac{6}{\theta^3} + \frac{2}{\theta^3} \right] \\
    &= 6 - 6 + 2 \\
    &= 2
\end{align*}

Le coefficient d'asymétrie théorique de la loi exponentielle de moyenne
1 est de 2. Cela s'avère compatible avec l'estimé ponctuel obtenu en a)
puisque celui-ci est de 1.3076743, ce qui est assez près de 2. Pour ce
qui est de l'intervalle de confiance obtenu en b), il est possible
d'observer que la valeur théorique est incluse dans cette intervalle qui
est 0.2089363, 2.4064123. Par conséquent, les estimateurs obtenus sont
compatibles avec la valeur théorique. Toutefois, ceux-ci ne sont pas
très précis en raison du nombre d'observations qui est assez faible,
soit 100 observations. En augmentant le nombre d'observation,
l'intervalle de confiance serait plus petit et l'estimateur serait plus
précis.

\newpage

\section{Question 2}\label{question-2}

\subsubsection{a) Estimation de l'espérance
limitée}\label{a-estimation-de-lesperance-limitee}

La fonction quantile théorique d'une loi exponentielle (\(\theta =1\))
est déterminée de cette façon :

\begin{align*}
    X 
    &\sim Exp(\theta = 1) \\
    F(x) 
    &= 1 - e^{-\frac{x}{\theta}} \\
    &= 1 - e^{-x} \\
    k 
    &= 1 - e^{-x} \\
    1 - k 
    &= e^{-x} \\
    F_x^{-1}(k) 
    &= x = -ln(1-k) \\
\end{align*}

\begin{table}[ht]
\centering
\begin{tabular}{cccc}
  \hline
 & Percentile & Limite u & Espérance limitée \\ 
  \hline
1 & 0.25 & 0.29 & 0.25 \\ 
  2 & 0.35 & 0.43 & 0.36 \\ 
  3 & 0.50 & 0.69 & 0.53 \\ 
  4 & 0.60 & 0.92 & 0.64 \\ 
  5 & 0.75 & 1.39 & 0.82 \\ 
  6 & 0.85 & 1.90 & 0.93 \\ 
   \hline
\end{tabular}
\caption{Valeurs de l'espérance limitée pour chaque limite u donnée} 
\end{table}

\(E[min(X,u)]\) estimé est une fonction croissante en fonction du \(u\).
En effet, le tableau présente une augmentation de la valeur de
l'espérance lorsque \(u\) augmente. Cela s'avère tout à fait logique
puisque lorsque \(u\) est petit, la fonction \(min(X,u)\) prend
davantage en considération les valeurs de \(u\). Par conséquent, les
valeurs supérieures de \(x\) sont réduites, ce qui réduit donc la
moyenne, car elle ne tient compte que des valeurs inférieures ou égales
à \(u\). Alors, si \(u\) augmente, l'espérance prendra en compte des
valeurs plus grande de \(x\), car \(u\) aura augmenté.

\subsubsection{b) Intervalle de confiance pour l'espérance
limitée}\label{b-intervalle-de-confiance-pour-lesperance-limitee}

\begin{table}[ht]
\centering
\begin{tabular}{ccccc}
  \hline
 & Percentile & Limite u & Espérance limité & Variance \\ 
  \hline
1 & 0.50 & 0.69 & 0.53 & 0.00 \\ 
  2 & 0.75 & 1.39 & 0.82 & 0.00 \\ 
   \hline
\end{tabular}
\caption{Valeurs de l'espérance limitée et de sa variance pour les percentiles 0.5 et 0.75} 
\end{table}

Avec la méthode de ré-échantillonnage, la variance estimée pour
\(E[min(X,F^{-1}(0.5))]\) est de 7.1668809\times 10\^{}\{-4\}, alors que
celle pour \(E[min(X,F^{-1}(0.75))]\) est de \(???????\). À partir des
estimateurs de \(E[min(X,u)]\) calculés en a) pour \(u = F^{-1}(0.5)\)
et \(u = F^{-1}(0.75)\) et de ces variances estimées, il est possible
d'obtenir les intervalles de confiance suivants :

\begin{align*}
[0.4740193, 0.5789598] ,\ pour \ u = F^{-1}(0.5)
\end{align*}\begin{align*}
[0.7124814, 0.9211603] ,\ pour \ u = F^{-1}(0.75)
\end{align*}

\newpage

\subsubsection{c) Espérance limitée
théorique}\label{c-esperance-limitee-theorique}

\begin{align*}
    X 
    &\sim Exp(\theta = 1) \\
    S(x) 
    &= e^{-x}, \ où \ x>0 \\
    u
    &=F_x^{-1}(k) = -ln(1-k)\\
    E\left[ min(X,u) \right]
    &=E\left[ X \wedge u \right] \\
    &=\int_{0}^{u} x f_x(x) dx + \int_{0}^{u} u f_x(x) dx \\
    &=\int_{0}^{u} S_x(x) dx \\
    &=\int_{0}^{u} e^{-x} dx  \\
    &=[-e^{-x}]_{0}^{u} \\
    &=1 - e^{-u} \\
    &=1 - e^{-(-ln(1-k))} \\
    &=1 - (1-k) \\
    &= k
\end{align*}

Ainsi, les valeurs théoriques des espérances limitées sont :

\begin{align*}
    E\left[min(X,F_x^{-1}(0.5))\right] = 0.5 \\
    E\left[min(X,F_x^{-1}(0.75))\right] = 0.75
\end{align*}

L'espérance limitée théorique de la loi exponentielle de moyenne 1 est
égale à son quantile. Par conséquent, \(E[min(X,F^{-1}(0.5))] = 0.5\) et
\(E[min(X,F^{-1}(0.75))] = 0.75\). Cela s'avère compatible avec les
valeurs estimées obtenues en a) puisque ceux-ci sont de 0.5264896 et
0.8168208, ce qui est assez près des valeurs théoriques. Pour ce qui est
des intervalles de confiance obtenu en b), il est possible d'observer
que les valeurs théoriques sont incluses dans chacun des intervalles
respectifs qui sont {[}0.4740193, 0.5789598{]}, pour le premier, et
{[}0.7124814, 0.9211603{]}, pour le second. Par conséquent, les
estimateurs obtenus sont compatibles avec les valeurs théoriques. Il est
donc possible d'affirmer que ces estimateurs non-paramétriques sont
assez performants puisque les valeurs obtenues sont très près de la
valeur théorique.

\newpage

\section{Question 3}\label{question-3}

\subsubsection{a) Détermination de la fonction de survie à l'aide de
l'estimateur
Kaplan-Meier}\label{a-determination-de-la-fonction-de-survie-a-laide-de-lestimateur-kaplan-meier}

\begin{table}[ht]
\centering
\begin{tabular}{cccccc}
  \hline
 & i & yi & Si & ri & Kaplan-Meier \\ 
  \hline
1 &   1 & 30.00 & 1.00 & 10.00 & 0.90 \\ 
  2 &   2 & 40.00 & 1.00 & 9.00 & 0.80 \\ 
  3 &   3 & 57.00 & 1.00 & 8.00 & 0.70 \\ 
  4 &   4 & 65.00 & 1.00 & 7.00 & 0.60 \\ 
  5 &   5 & 84.00 & 1.00 & 5.00 & 0.48 \\ 
  6 &   6 & 90.00 & 1.00 & 4.00 & 0.36 \\ 
  7 &   7 & 98.00 & 1.00 & 2.00 & 0.18 \\ 
  8 &   8 & 101.00 & 1.00 & 1.00 & 0.00 \\ 
   \hline
\end{tabular}
\caption{Données permettant de trouver l'estimateur Kaplan-Meier} 
\end{table}

La table précèdente présente toutes les données utiles permettant de
déterminer l'estimateur Kaplan-Meier. Par conséquent, il est possible de
calculer cet estimateur à partir de sa définition :

\begin{align*}
    \hat{S}_n(t) &= \left\{
        \begin{array}{lll}
            1, \ 0\leqslant t<y,\\
            \\
            \prod\limits_{i=1}^{j-1}(\frac{r_i-S_i}{r_i})  ,\ y_{j-1}\leqslant t<y_i,\\
            \\
            \prod\limits_{i=1}^{k}(\frac{r_i-S_i}{r_i}) , \ t>y_k
        \end{array}
    \right. \\
    \hat{S}_n(i) &= \left\{
        \begin{array}{ll}
            1                   &, 0\leqslant i<1,\\
            0.9&, 1\leqslant i<2,\\
            0.8&, 2\leqslant i<3,\\
            0.7&, 3\leqslant i<4,\\
            0.6&, 4\leqslant i<5,\\
            0.48&, 5\leqslant i<6,\\
            0.36&, 6\leqslant i<7,\\
            0.18&, 7\leqslant i<8,\\
            0&, i \geqslant 8
        \end{array}
    \right.
\end{align*}

\subsubsection{\texorpdfstring{b) Graphique de l'estimateur Kaplan-Meier
et intervalle de confiance pour
\(S_n(50)\)}{b) Graphique de l'estimateur Kaplan-Meier et intervalle de confiance pour S\_n(50)}}\label{b-graphique-de-lestimateur-kaplan-meier-et-intervalle-de-confiance-pour-s_n50}

\includegraphics{Rapport_tp1_files/figure-latex/unnamed-chunk-6-1.pdf}

Calculer un estimateur pour la variance de la fonction de survie à
l'aide de la formule de Greenwood :

\begin{align*}
    \widehat{Var}(\hat{S}_n(t))
    &=[\hat{S}_n(y_j)]^2\sum_{i=1}^j\frac{S_i}{r_i(S_i-r_i)} \\
    \widehat{Var}(\hat{S}_n(y_j)) &= \left\{
        \begin{array}{ll}
            0.009&,\ i=1,\\
            0.016&,\ i=2,\\
            0.021&,\ i=3,\\
            0.024&,\ i=4,\\
            0.02688&,\ i=5,\\
            0.02592&,\ i=6,\\
            0.02268&,\ i=7
        \end{array}
    \right.
\end{align*}

Alors, en ayant les valeurs de l'estimateur Kaplan-Meier et celles de
Greenwood, il est possible de déduire les intervalles de confiance au
moment des décès. Toutefois, il faut déterminer l'intervalle de
confiance au niveau 95\% pour S(50), qui est entre deux moments de
décès. Alors, il faut faire une interpolation linéaire (OGIVE) entre
\(y_i = 40\) et \(y_i = 57\).

\newpage

La fonction de survie sous OGIVE est donnée par:

\begin{align*}
    F_n^{OGIVE}(x)
    &= (\frac{c_j-x}{c_j-c_{j-1}}) F_n(c_{j-1}) + (\frac{x-c_{j-1}}{c_j-c_{j-1}}) F_n(c_{j}) \\
    S_n^{OGIVE}(x)
    &= 1-F_n^{OGIVE}(x) \\
    &= 1- (\frac{c_j-x}{c_j-c_{j-1}}) F_n(c_{j-1}) - (\frac{x-c_{j-1}}{c_j-c_{j-1}}) F_n(c_{j}) \\
    &= 1 - (\frac{c_j-x}{c_j-c_{j-1}}) + (\frac{c_j-x}{c_j-c_{j-1}}) S_n(c_{j-1}) - (\frac{x-c_{j-1}}{c_j-c_{j-1}}) + (\frac{x-c_{j-1}}{c_j-c_{j-1}}) S_n(c_{j}) \\
    &= 1- (\frac{(c_j-x) + (x-c_{j-1})}{c_j-c_{j-1}}) + (\frac{c_j-x}{c_j-c_{j-1}}) S_n(c_{j-1}) - (\frac{x-c_{j-1}}{c_j-c_{j-1}}) S_n(c_{j}) \\
    &= 1-1+(\frac{c_j-x}{c_j-c_{j-1}}) S_n(c_{j-1}) - (\frac{x-c_{j-1}}{c_j-c_{j-1}}) S_n(c_{j}) \\
    &=(\frac{c_j-x}{c_j-c_{j-1}}) S_n(c_{j-1}) - (\frac{x-c_{j-1}}{c_j-c_{j-1}}) S_n(c_{j})
\end{align*}

Donc, dans le cas de \(S_n(50)\), la fonction de survie sous OGIVE est :

\begin{align*}
    S_n^{OGIVE}(50)
    &=(\frac{57-50}{57-40}) S_n(y_{40}) - (\frac{50-40}{57-40}) S_n(y_{57}) \\
    &= 0.7411765
\end{align*}

L'estimer de la variance sous OGIVE est donnée par:

\begin{align*}
    Var(S_n^{OGIVE}(x))
    &= \frac{(c_j-c_{j-1})^2Var(Y) + (x-c_{j-1})^2Var(Z) + 2(c_j-c_{j-1})(x-c_{j-1})Cov(Y,Z)}{(n(c_j-c_{j-1}))^2} 
\end{align*}

où,

\begin{align*}
    Var(Y)
    &= n S_n(c_{j-1}) (1 - S_n(c_{j-1})) \\
    Var(Z)
    &= n (S_n(c_{j-1}) - S_n(c_{j}) (1 - S_n(c_{j-1}) + S_n(c_{j}))) \\
    Cov(Y,Z)
    &= -n (1 - S_n(c_{j-1})) (S_n(c_{j-1}) - S_n(c_{j}))
\end{align*}

Donc, dans le cas de \(S_n(50)\), la variance de la fonction de survie
sous OGIVE est :

\begin{align*}
    Var(Y) 
    &= 8 S_n(y_{40}) (1 - S_n(y_{40})) = 1.28 \\
    Var(Z)
    &= 8 (S_n(y_{40}) - S_n(y_{57}) (1 - S_n(y_{40}) + S_n(y_{57}))) = 1.36\\
    Cov(Y,Z)
    &= -8 (1 - S_n(y_{40})) (S_n(y_{40}) - S_n(y_{57})) = -0.16 \\
    \\
    Var(S_n^{OGIVE}(50))
    &= \frac{(57-40)^2Var(Y) + (50-40)^2Var(Z) + 2(57-40)(50-40)Cov(Y,Z)}{(8(57-40))^2} \\
    &= 0.0244118
\end{align*}

\newpage

Ainsi, en ayant \(S_n(50)\) et \(Var(S_n(50))\), il est possible
d'évaluer l'intervalle de confiance au niveau de confiance 95\%.

\begin{align*}
    IC \ pour \ S_n(y_{50})
    &: \ \hat{S}_n(y_{50})) \pm z_{0.975} \sqrt{\widehat{Var}(\hat{S}_n(y_{50}))}\\
    &: \ 0.7411765 \pm z_{0.975} \sqrt{0.0244118} \\
    &: \ [0.4349465, 1.0474064]
\end{align*}

\section{Mettre un commentaire ici}\label{mettre-un-commentaire-ici}

\subsubsection{\texorpdfstring{c) Intervalle de confiance pour
\(S_n(50)\) avec la transformation log
(-log)}{c) Intervalle de confiance pour S\_n(50) avec la transformation log (-log)}}\label{c-intervalle-de-confiance-pour-s_n50-avec-la-transformation-log--log}

La valeur estimée de la fonction de survie et la valeur estimée de la
variance sont toujours les mêmes que celles de la section b). Par
conséquent, seulement l'intervalle de confiance est modifié. Alors,
l'intervalle de confiance log (-log) au niveau de confiance 95\% est
déterminé de cette façon :

\begin{align*}
    IC \ log \ transformed \ pour \ S_n(y_{50})
    &: \ [\hat{S}_n(y_{50}))^{1/u} , \hat{S}_n(y_{50}))^{u}] \\
    où \ u
    &= e^{\frac{z_{0.975} \sqrt{\hat{Var}(\hat{S}_n(y_{50}))}}{\hat{S}_n(y_{50})) ln(\hat{S}_n(y_{50})))}} \\
    &= e^{\frac{z_{0.975} \sqrt{0.0244118}}{0.7411765 ln(0.7411765)}} \\
    \\
    IC \ log \ transformed \ pour \ S_n(y_{50}) 
    &: \ [0.3042543, 0.9273784]
\end{align*}

\section{Mettre commentaire ici}\label{mettre-commentaire-ici}

L'intervalle de confiance devrait être plus petit et précis dans le cas
de la transformation log.

\newpage

\section{Annexe}\label{annexe}

\section{Question 1}\label{question-1-1}

\subsubsection{a) Estimation du coefficient
d'asymétrie}\label{a-estimation-du-coefficient-dasymetrie-1}

Échantillon de 100 données simulées à partir d'une loi exponentielle
(\(\theta =1\)) :

\begin{Shaded}
\begin{Highlighting}[]
\NormalTok{data <-}\StringTok{  }\KeywordTok{rexp}\NormalTok{(}\DecValTok{100}\NormalTok{, }\DecValTok{1}\NormalTok{)}
\end{Highlighting}
\end{Shaded}

\begin{verbatim}
##   [1] 1.045035824 0.462386753 2.750051236 2.291154062 0.033701140
##   [6] 0.615074305 2.571114166 1.609049067 1.887332367 0.140717380
##  [11] 0.154829331 1.635541158 0.590173881 1.628642179 2.760963892
##  [16] 0.038533552 0.402069510 0.889676860 4.756520260 0.109648435
##  [21] 0.430263955 0.792066487 0.677247164 1.282415408 0.065872191
##  [26] 0.802996605 2.557728389 2.240621588 1.950693346 0.649358110
##  [31] 0.777562920 1.366246748 0.936110420 1.058525182 0.602101246
##  [36] 1.254954812 0.989935359 0.901897225 0.153660473 1.515543282
##  [41] 0.270008921 0.595027370 0.489034373 0.420450607 1.130850831
##  [46] 2.776169524 1.692107839 1.317417553 0.123528257 3.976053369
##  [51] 0.955571028 1.110779357 0.042962004 1.425330060 0.151452010
##  [56] 0.606168081 1.261892802 0.826618979 0.538590088 0.145382218
##  [61] 0.520546786 1.204000278 0.139570961 0.044160222 0.382780013
##  [66] 0.002100709 1.725137759 0.306018017 0.984789335 1.266538891
##  [71] 2.485354459 0.344354076 1.891666884 0.450025352 1.279359689
##  [76] 0.046669251 0.342278319 2.682279793 1.355412098 0.338098234
##  [81] 2.129145985 0.267059965 0.283757456 0.868358850 0.869229000
##  [86] 0.121072161 1.011644976 3.272451751 1.737017207 2.162735793
##  [91] 0.759573730 0.085831038 0.817995727 0.217807683 0.539964824
##  [96] 1.259277469 1.718162406 0.199231977 2.326852387 0.735832595
\end{verbatim}

\begin{Shaded}
\begin{Highlighting}[]
\CommentTok{# Fonction pour calculer le coefficient d'asymétrie.}
\NormalTok{coef_asymetrie <-}\StringTok{ }\ControlFlowTok{function}\NormalTok{(x)\{}
\NormalTok{    mu <-}\StringTok{ }\KeywordTok{mean}\NormalTok{(x)}
\NormalTok{    sd <-}\StringTok{ }\KeywordTok{sd}\NormalTok{(x)}
    \KeywordTok{mean}\NormalTok{((x }\OperatorTok{-}\StringTok{ }\NormalTok{mu)}\OperatorTok{^}\DecValTok{3}\NormalTok{) }\OperatorTok{/}\StringTok{ }\NormalTok{sd}\OperatorTok{^}\DecValTok{3} 
\NormalTok{\}}

\CommentTok{# Estimation du coefficient d'asymétrie des données.}
\NormalTok{estimateur_coef_asymetrie <-}\StringTok{ }\KeywordTok{coef_asymetrie}\NormalTok{(data)}

\CommentTok{# Histograme des données avec la courbe théorique d'un loi exponentielle (teta = 1)}
\KeywordTok{hist}\NormalTok{(data, }\DataTypeTok{probability =} \OtherTok{TRUE}\NormalTok{,}
     \DataTypeTok{main =} \StringTok{"Histogramme des données"}\NormalTok{,}
     \DataTypeTok{ylab =} \StringTok{"Densité"}\NormalTok{,}
     \DataTypeTok{xlab =} \StringTok{"Données"}\NormalTok{);}\KeywordTok{curve}\NormalTok{(}\KeywordTok{dexp}\NormalTok{(x,}\DecValTok{1}\NormalTok{), }\DataTypeTok{add =} \OtherTok{TRUE}\NormalTok{)}

\CommentTok{# Histograme des données avec la courbe théorique d'un loi exponentielle (teta = 1)}
\KeywordTok{library}\NormalTok{(ggplot2)}
\NormalTok{data2 <-}\StringTok{ }\KeywordTok{data.frame}\NormalTok{(data)}
\NormalTok{df <-}\StringTok{ }\KeywordTok{data.frame}\NormalTok{(}\DataTypeTok{x =}\NormalTok{ data, }\DataTypeTok{y =} \KeywordTok{dexp}\NormalTok{(data, }\DecValTok{1}\NormalTok{))}

\KeywordTok{ggplot}\NormalTok{(}\DataTypeTok{data =}\NormalTok{ data2) }\OperatorTok{+}
\StringTok{        }\KeywordTok{geom_histogram}\NormalTok{(}\KeywordTok{aes}\NormalTok{(}\DataTypeTok{x =}\NormalTok{ data, }\DataTypeTok{y =}\NormalTok{ ..density..),}
                   \DataTypeTok{binwidth =} \FloatTok{0.25}\NormalTok{, }\DataTypeTok{fill =} \StringTok{"grey"}\NormalTok{, }\DataTypeTok{color =} \StringTok{"black"}\NormalTok{) }\OperatorTok{+}
\StringTok{        }\KeywordTok{geom_line}\NormalTok{(}\DataTypeTok{data =}\NormalTok{ df, }\KeywordTok{aes}\NormalTok{(}\DataTypeTok{x =}\NormalTok{ data, }\DataTypeTok{y =}\NormalTok{ y), }\DataTypeTok{color =} \StringTok{"red"}\NormalTok{) }\OperatorTok{+}
\StringTok{        }\KeywordTok{ggtitle}\NormalTok{(}\StringTok{"Histogramme des valeurs simulées à partir }
\StringTok{                    d'une loi exponentielle de moyenne 1"}\NormalTok{) }\OperatorTok{+}\StringTok{ }
\StringTok{        }\KeywordTok{theme}\NormalTok{(}\DataTypeTok{plot.title =}  \KeywordTok{element_text}\NormalTok{(}\DataTypeTok{face=}\StringTok{"bold"}\NormalTok{, }\DataTypeTok{hjust =} \FloatTok{0.5}\NormalTok{)) }\OperatorTok{+}
\StringTok{        }\KeywordTok{xlab}\NormalTok{(}\StringTok{"Données"}\NormalTok{) }\OperatorTok{+}
\StringTok{        }\KeywordTok{ylab}\NormalTok{(}\StringTok{"Densité"}\NormalTok{) }\OperatorTok{+}
\StringTok{        }\KeywordTok{scale_y_continuous}\NormalTok{(}\DataTypeTok{limits =} \KeywordTok{c}\NormalTok{(}\DecValTok{0}\NormalTok{, }\DecValTok{1}\NormalTok{))}
\end{Highlighting}
\end{Shaded}

\subsubsection{b) Intervalle de confiance pour le coefficient
d'asymétrie}\label{b-intervalle-de-confiance-pour-le-coefficient-dasymetrie-1}

\begin{Shaded}
\begin{Highlighting}[]
\CommentTok{# Estimation du theta }
\NormalTok{teta <-}\StringTok{ }\KeywordTok{mean}\NormalTok{(data)}

\CommentTok{# Simulation de 50 échantillions}
\NormalTok{echantillion <-}\StringTok{ }\KeywordTok{lapply}\NormalTok{(}\DecValTok{1}\OperatorTok{:}\DecValTok{50}\NormalTok{, }\ControlFlowTok{function}\NormalTok{(i) }\KeywordTok{rexp}\NormalTok{(}\DecValTok{100}\NormalTok{, teta))}

\CommentTok{# Estimation des 50 coefficients d'asymétrie}
\NormalTok{coef_asymetrie_simul <-}\StringTok{ }\KeywordTok{sapply}\NormalTok{(}\DecValTok{1}\OperatorTok{:}\DecValTok{50}\NormalTok{, }\ControlFlowTok{function}\NormalTok{(i) }\KeywordTok{coef_asymetrie}\NormalTok{(echantillion[[i]]) )}

\CommentTok{# Estimation de la variance empirique}
\NormalTok{variance_coef_asymetrie <-}\StringTok{ }\KeywordTok{var}\NormalTok{(coef_asymetrie_simul)}

\CommentTok{# Interval de confiance pour le coefficient d'asymétrie}
\NormalTok{IC_coef_asymetrie <-}\StringTok{ }\KeywordTok{cbind}\NormalTok{(estimateur_coef_asymetrie }\OperatorTok{-}\StringTok{ }
\StringTok{                               }\KeywordTok{qnorm}\NormalTok{(}\FloatTok{0.975}\NormalTok{) }\OperatorTok{*}\StringTok{ }\KeywordTok{sqrt}\NormalTok{(variance_coef_asymetrie),}
\NormalTok{                           estimateur_coef_asymetrie }\OperatorTok{+}\StringTok{ }
\StringTok{                               }\KeywordTok{qnorm}\NormalTok{(}\FloatTok{0.975}\NormalTok{) }\OperatorTok{*}\StringTok{ }\KeywordTok{sqrt}\NormalTok{(variance_coef_asymetrie))}
\end{Highlighting}
\end{Shaded}

\subsubsection{c) Coefficient d'asymétrie
théorique}\label{c-coefficient-dasymetrie-theorique-1}

Aucun calcul R n'a été fait dans cette section.

\newpage

\section{Question 2}\label{question-2-1}

\subsubsection{\texorpdfstring{a) Estimation de
\(E[min(X,u)]\)}{a) Estimation de E{[}min(X,u){]}}}\label{a-estimation-de-eminxu}

\begin{Shaded}
\begin{Highlighting}[]
\CommentTok{# Déterminer les limites u à l'aide de la fonction quantile}
\NormalTok{k <-}\StringTok{ }\KeywordTok{c}\NormalTok{(}\FloatTok{0.25}\NormalTok{, }\FloatTok{0.35}\NormalTok{, }\FloatTok{0.5}\NormalTok{, }\FloatTok{0.6}\NormalTok{, }\FloatTok{0.75}\NormalTok{, }\FloatTok{0.85}\NormalTok{)}
\NormalTok{fonction_quantile <-}\StringTok{ }\ControlFlowTok{function}\NormalTok{(x) }\OperatorTok{-}\KeywordTok{log}\NormalTok{(}\DecValTok{1}\OperatorTok{-}\NormalTok{x)}
\NormalTok{limite_u <-}\StringTok{ }\KeywordTok{sapply}\NormalTok{(k, }\ControlFlowTok{function}\NormalTok{(i) }\KeywordTok{fonction_quantile}\NormalTok{(i))}

\CommentTok{# Déterminer l'espérance limitée pour chacune des limites u}
\NormalTok{Estimateur_Esperance_limite <-}\StringTok{ }\KeywordTok{sapply}\NormalTok{(}\DecValTok{1}\OperatorTok{:}\KeywordTok{length}\NormalTok{(limite_u), }\ControlFlowTok{function}\NormalTok{(u) }
                                    \KeywordTok{mean}\NormalTok{(}\KeywordTok{sapply}\NormalTok{(}\DecValTok{1}\OperatorTok{:}\DecValTok{100}\NormalTok{, }\ControlFlowTok{function}\NormalTok{(i) }
                                        \KeywordTok{min}\NormalTok{(data[i], limite_u[u]))))  }

\CommentTok{# Publier les résultats pour chaque u}
\NormalTok{resultats <-}\StringTok{ }\KeywordTok{data.frame}\NormalTok{(k, limite_u, Estimateur_Esperance_limite)}
\KeywordTok{colnames}\NormalTok{(resultats) <-}\StringTok{ }\KeywordTok{c}\NormalTok{(}\StringTok{'Percentile'}\NormalTok{,}\StringTok{"Limite u"}\NormalTok{,}\StringTok{"Espérance limitée"}\NormalTok{)}

\KeywordTok{library}\NormalTok{(xtable)}
\KeywordTok{options}\NormalTok{(}\DataTypeTok{xtable.comment =} \OtherTok{FALSE}\NormalTok{)}
\KeywordTok{xtable}\NormalTok{(resultats, }\DataTypeTok{caption =} \StringTok{"Valeurs de l'espérance limitée pour chaque limite u donnée"}\NormalTok{,  }
                 \DataTypeTok{align =} \KeywordTok{c}\NormalTok{(}\StringTok{"c"}\NormalTok{, }\StringTok{"c"}\NormalTok{, }\StringTok{"c"}\NormalTok{, }\StringTok{"c"}\NormalTok{)) }
\end{Highlighting}
\end{Shaded}

\subsubsection{\texorpdfstring{b) Intervalle de confiance pour
\(E[min(X,u)]\)}{b) Intervalle de confiance pour E{[}min(X,u){]}}}\label{b-intervalle-de-confiance-pour-eminxu}

\begin{Shaded}
\begin{Highlighting}[]
\CommentTok{# Estimer le paramètre teta de la loi exponentielle}
\NormalTok{teta <-}\StringTok{ }\KeywordTok{mean}\NormalTok{(data)}

\CommentTok{# Créer 50 échantillons de 100 données}
\NormalTok{echantillion <-}\StringTok{ }\KeywordTok{lapply}\NormalTok{(}\DecValTok{1}\OperatorTok{:}\DecValTok{50}\NormalTok{, }\ControlFlowTok{function}\NormalTok{(i) }\KeywordTok{rexp}\NormalTok{(}\DecValTok{100}\NormalTok{, teta))}

\CommentTok{# Déterminer l'espérance limitée pour chacun des 50 échantillons et pour }
\CommentTok{#les percentiles 0.5 et 0.75}
\NormalTok{Esperance_limite_simul <-}\StringTok{ }\KeywordTok{sapply}\NormalTok{(}\KeywordTok{c}\NormalTok{(}\DecValTok{3}\NormalTok{,}\DecValTok{5}\NormalTok{), }\ControlFlowTok{function}\NormalTok{(u) }
                                \KeywordTok{sapply}\NormalTok{(}\DecValTok{1}\OperatorTok{:}\DecValTok{50}\NormalTok{, }\ControlFlowTok{function}\NormalTok{(j) }
                                    \KeywordTok{mean}\NormalTok{(}\KeywordTok{sapply}\NormalTok{(}\DecValTok{1}\OperatorTok{:}\DecValTok{100}\NormalTok{, }\ControlFlowTok{function}\NormalTok{(i) }
                                        \KeywordTok{min}\NormalTok{(echantillion[[j]][i], limite_u[u])))))}

\CommentTok{# Déterminer la variance de l'espérance limitée pour les deux percentiles donnés}
\NormalTok{variance_Esperance_limite_simul <-}\StringTok{ }\KeywordTok{sapply}\NormalTok{(}\DecValTok{1}\OperatorTok{:}\DecValTok{2}\NormalTok{,}\ControlFlowTok{function}\NormalTok{(i) }\KeywordTok{var}\NormalTok{(Esperance_limite_simul[,i]))}

\CommentTok{# Déterminer l'intervalle de confiance}
\NormalTok{Est_Esperance_limite <-}\StringTok{ }\KeywordTok{rep}\NormalTok{(}\DecValTok{0}\NormalTok{,}\DecValTok{2}\NormalTok{)}
\NormalTok{Est_Esperance_limite[}\DecValTok{1}\NormalTok{] <-}\StringTok{ }\NormalTok{Estimateur_Esperance_limite[}\DecValTok{3}\NormalTok{]}
\NormalTok{Est_Esperance_limite[}\DecValTok{2}\NormalTok{] <-}\StringTok{ }\NormalTok{Estimateur_Esperance_limite[}\DecValTok{5}\NormalTok{]}

\NormalTok{IC_Esperance_limite_simul <-}\StringTok{ }\KeywordTok{lapply}\NormalTok{(}\DecValTok{1}\OperatorTok{:}\DecValTok{2}\NormalTok{, }\ControlFlowTok{function}\NormalTok{(i) }
                                \KeywordTok{cbind}\NormalTok{(Est_Esperance_limite[i] }\OperatorTok{-}\StringTok{ }\KeywordTok{qnorm}\NormalTok{(}\FloatTok{0.975}\NormalTok{) }\OperatorTok{*}\StringTok{ }
\StringTok{                                          }\KeywordTok{sqrt}\NormalTok{(variance_Esperance_limite_simul[i]),}
\NormalTok{                                    Est_Esperance_limite[i] }\OperatorTok{+}\StringTok{ }\KeywordTok{qnorm}\NormalTok{(}\FloatTok{0.975}\NormalTok{) }\OperatorTok{*}
\StringTok{                                        }\KeywordTok{sqrt}\NormalTok{(variance_Esperance_limite_simul[i])))}

\CommentTok{# Publier les résultats }
\NormalTok{resultats_}\DecValTok{2}\NormalTok{ <-}\StringTok{ }\KeywordTok{data.frame}\NormalTok{(}\KeywordTok{c}\NormalTok{(k[}\DecValTok{3}\NormalTok{], k[}\DecValTok{5}\NormalTok{]), }\KeywordTok{c}\NormalTok{(limite_u[}\DecValTok{3}\NormalTok{], limite_u[}\DecValTok{5}\NormalTok{]), }
\NormalTok{                          Est_Esperance_limite, variance_Esperance_limite_simul)}
\KeywordTok{colnames}\NormalTok{(resultats_}\DecValTok{2}\NormalTok{) <-}\StringTok{ }\KeywordTok{c}\NormalTok{(}\StringTok{"Percentile"}\NormalTok{,}\StringTok{"Limite u"}\NormalTok{,}\StringTok{"Espérance limité"}\NormalTok{, }\StringTok{"Variance"}\NormalTok{)}

\KeywordTok{library}\NormalTok{(xtable)}
\KeywordTok{options}\NormalTok{(}\DataTypeTok{xtable.comment =} \OtherTok{FALSE}\NormalTok{)}
\KeywordTok{xtable}\NormalTok{(resultats_}\DecValTok{2}\NormalTok{, }\DataTypeTok{caption =} \StringTok{"Valeurs de l'espérance limitée et de sa variance }
\StringTok{                 pour les percentiles 0.5 et 0.75"}\NormalTok{,  }
                 \DataTypeTok{align =} \KeywordTok{c}\NormalTok{(}\StringTok{"c"}\NormalTok{, }\StringTok{"c"}\NormalTok{, }\StringTok{"c"}\NormalTok{, }\StringTok{"c"}\NormalTok{, }\StringTok{"c"}\NormalTok{))}
\end{Highlighting}
\end{Shaded}

\subsubsection{\texorpdfstring{c) \(E[min(X,u)]\)
théorique}{c) E{[}min(X,u){]} théorique}}\label{c-eminxu-theorique}

Aucun calcul R n'a été fait dans cette section.

\newpage

\section{Question 3}\label{question-3-1}

\subsubsection{a) Détermination de la fonction de survie à l'aide de
l'estimateur
Kaplan-Meier}\label{a-determination-de-la-fonction-de-survie-a-laide-de-lestimateur-kaplan-meier-1}

\begin{Shaded}
\begin{Highlighting}[]
\CommentTok{# Présentation des données du problème}
\NormalTok{tableau1 <-}\StringTok{ }\NormalTok{\{}
\NormalTok{    Temps <-}\StringTok{ }\KeywordTok{c}\NormalTok{(}\DecValTok{30}\NormalTok{, }\DecValTok{40}\NormalTok{, }\DecValTok{57}\NormalTok{, }\DecValTok{65}\NormalTok{, }\DecValTok{65}\NormalTok{, }\DecValTok{84}\NormalTok{, }\DecValTok{90}\NormalTok{, }\DecValTok{92}\NormalTok{, }\DecValTok{98}\NormalTok{, }\DecValTok{101}\NormalTok{) }
\NormalTok{    Cens <-}\StringTok{ }\KeywordTok{c}\NormalTok{(}\DecValTok{1}\NormalTok{, }\DecValTok{1}\NormalTok{, }\DecValTok{1}\NormalTok{, }\DecValTok{1}\NormalTok{, }\DecValTok{0}\NormalTok{, }\DecValTok{1}\NormalTok{, }\DecValTok{1}\NormalTok{, }\DecValTok{0}\NormalTok{, }\DecValTok{1}\NormalTok{, }\DecValTok{1}\NormalTok{)}

    \KeywordTok{data.frame}\NormalTok{(Temps, Cens)}
\NormalTok{\} }

\CommentTok{# Tableau détaillé permettant de calculer l'estimateur Kaplan-Meier}
\NormalTok{tableau2 <-}\StringTok{ }\NormalTok{\{}
\NormalTok{    yi <-}\StringTok{ }\KeywordTok{unique}\NormalTok{(tableau1[}\KeywordTok{which}\NormalTok{(tableau1[,}\DecValTok{2}\NormalTok{] }\OperatorTok{!=}\StringTok{ }\DecValTok{0}\NormalTok{),}\DecValTok{1}\NormalTok{])    }\CommentTok{# moments uniques des décès}
\NormalTok{    i <-}\StringTok{ }\DecValTok{1}\OperatorTok{:}\KeywordTok{length}\NormalTok{(yi) }
\NormalTok{    Si <-}\StringTok{ }\KeywordTok{rep}\NormalTok{(}\DecValTok{1}\NormalTok{,}\KeywordTok{length}\NormalTok{(yi))                             }\CommentTok{# nombre de décès au temps yi}
\NormalTok{    ri <-}\StringTok{ }\KeywordTok{c}\NormalTok{(}\DecValTok{10}\NormalTok{,}\DecValTok{9}\NormalTok{,}\DecValTok{8}\NormalTok{,}\DecValTok{7}\NormalTok{,}\DecValTok{5}\NormalTok{,}\DecValTok{4}\NormalTok{,}\DecValTok{2}\NormalTok{,}\DecValTok{1}\NormalTok{)                           }\CommentTok{# nombre de survivants au temps yi}
    
    \KeywordTok{data.frame}\NormalTok{(i, yi, Si, ri, }\StringTok{"Kaplan Meier"}\NormalTok{ =}\StringTok{ }\KeywordTok{cumprod}\NormalTok{(}\DecValTok{1}\OperatorTok{-}\NormalTok{Si}\OperatorTok{/}\NormalTok{ri))}
\NormalTok{\}}
\KeywordTok{colnames}\NormalTok{(tableau2) <-}\StringTok{ }\KeywordTok{c}\NormalTok{(}\StringTok{"i"}\NormalTok{, }\StringTok{"yi"}\NormalTok{, }\StringTok{"Si"}\NormalTok{, }\StringTok{"ri"}\NormalTok{, }\StringTok{"Kaplan-Meier"}\NormalTok{)}

\CommentTok{# Déterminer l'estimateur Kaplan-Meier pour chacun des moments uniques des décès}
\NormalTok{Estimateur_KM <-}\StringTok{ }\KeywordTok{cumprod}\NormalTok{(}\DecValTok{1}\OperatorTok{-}\NormalTok{Si}\OperatorTok{/}\NormalTok{ri)}

\CommentTok{# Publier le tableau 2 }
\KeywordTok{library}\NormalTok{(xtable)}
\KeywordTok{options}\NormalTok{(}\DataTypeTok{xtable.comment =} \OtherTok{FALSE}\NormalTok{)}
\KeywordTok{xtable}\NormalTok{(tableau2, }\DataTypeTok{caption =} \StringTok{"Données permettant de trouver l'estimateur Kaplan-Meier"}\NormalTok{,  }
                 \DataTypeTok{align =} \KeywordTok{c}\NormalTok{(}\StringTok{"c"}\NormalTok{, }\StringTok{"c"}\NormalTok{, }\StringTok{"c"}\NormalTok{, }\StringTok{"c"}\NormalTok{, }\StringTok{"c"}\NormalTok{, }\StringTok{"c"}\NormalTok{)) }
\end{Highlighting}
\end{Shaded}

\subsubsection{\texorpdfstring{b) Graphique de l'estimateur Kaplan-Meier
et intervalle de confiance pour
\(S_n(50)\)}{b) Graphique de l'estimateur Kaplan-Meier et intervalle de confiance pour S\_n(50)}}\label{b-graphique-de-lestimateur-kaplan-meier-et-intervalle-de-confiance-pour-s_n50-1}

\begin{Shaded}
\begin{Highlighting}[]
\CommentTok{# Graphique de la fonction de survie estimée par l'estimateur Kaplan-Meier}
\KeywordTok{library}\NormalTok{(ggplot2)}
\KeywordTok{ggplot}\NormalTok{(}\DataTypeTok{data =}\NormalTok{ tableau2, }\KeywordTok{aes}\NormalTok{(}\DataTypeTok{x=}\KeywordTok{unlist}\NormalTok{(tableau2[}\StringTok{"yi"}\NormalTok{]),}\DataTypeTok{y=}\KeywordTok{unlist}\NormalTok{(tableau2[}\StringTok{"Kaplan-Meier"}\NormalTok{])))}\OperatorTok{+}\StringTok{ }
\StringTok{        }\KeywordTok{geom_point}\NormalTok{() }\OperatorTok{+}\StringTok{ }
\StringTok{        }\KeywordTok{geom_line}\NormalTok{() }\OperatorTok{+}\StringTok{ }
\StringTok{        }\KeywordTok{ggtitle}\NormalTok{(}\StringTok{"Valeur de la fonction de survie estimée à l'aide de l'estimateur}
\StringTok{                Kaplan-Meier en fonction des temps de décès ou de retraits"}\NormalTok{) }\OperatorTok{+}\StringTok{ }
\StringTok{        }\KeywordTok{theme}\NormalTok{(}\DataTypeTok{plot.title =}  \KeywordTok{element_text}\NormalTok{(}\DataTypeTok{face=}\StringTok{"bold"}\NormalTok{, }\DataTypeTok{hjust =} \FloatTok{0.5}\NormalTok{)) }\OperatorTok{+}
\StringTok{        }\KeywordTok{xlab}\NormalTok{(}\StringTok{"Temps"}\NormalTok{) }\OperatorTok{+}
\StringTok{        }\KeywordTok{ylab}\NormalTok{(}\StringTok{"Estimateur Kaplan-Meier"}\NormalTok{)}

\CommentTok{# Calculer un estimateur pour la variance de la fonction de survie à l'aide de }
\CommentTok{# la formule de Greenwood}
\NormalTok{formule_Greenwood <-}\StringTok{ }\NormalTok{Estimateur_KM}\OperatorTok{^}\DecValTok{2} \OperatorTok{*}\StringTok{ }\KeywordTok{cumsum}\NormalTok{(Si}\OperatorTok{/}\NormalTok{ri}\OperatorTok{/}\NormalTok{(ri}\OperatorTok{-}\NormalTok{Si))}

\CommentTok{# Pour ce qui est de l'intervalle de confiance au niveau 95% pour S(50), il faut faire}
\CommentTok{# une interpolation linéaire (OGIVE) entre y_i = 40 et y_i = 57.}

\CommentTok{# Estimer S(50) avec la fonction OGIVE}
\NormalTok{Sn_}\DecValTok{50}\NormalTok{ <-}\StringTok{ }\NormalTok{\{}
\NormalTok{    x <-}\StringTok{ }\DecValTok{50}
\NormalTok{    ci <-}\StringTok{ }\DecValTok{40}
\NormalTok{    cj <-}\StringTok{ }\DecValTok{57}
\NormalTok{    alpha <-}\StringTok{ }\NormalTok{(cj}\OperatorTok{-}\NormalTok{x)}\OperatorTok{/}\NormalTok{(cj}\OperatorTok{-}\NormalTok{ci)}
    
\NormalTok{    Estimateur_KM[}\DecValTok{2}\NormalTok{] }\OperatorTok{*}\StringTok{ }\NormalTok{alpha }\OperatorTok{+}\StringTok{ }\NormalTok{Estimateur_KM[}\DecValTok{3}\NormalTok{] }\OperatorTok{*}\StringTok{ }\NormalTok{(}\DecValTok{1} \OperatorTok{-}\StringTok{ }\NormalTok{alpha) }
\NormalTok{\}}

\CommentTok{# Estimer Var(S(50)) avec la fonction OGIVE}
\NormalTok{Var_Sn_}\DecValTok{50}\NormalTok{ <-}\StringTok{ }\NormalTok{\{}
\NormalTok{    x <-}\StringTok{ }\DecValTok{50}
\NormalTok{    ci <-}\StringTok{ }\DecValTok{40}
\NormalTok{    cj <-}\StringTok{ }\DecValTok{57}
\NormalTok{    alpha <-}\StringTok{ }\NormalTok{(cj}\OperatorTok{-}\NormalTok{x)}\OperatorTok{/}\NormalTok{(cj}\OperatorTok{-}\NormalTok{ci)}
\NormalTok{    n <-}\StringTok{ }\KeywordTok{length}\NormalTok{(yi)}
    
\NormalTok{    var_Y <-}\StringTok{ }\NormalTok{n }\OperatorTok{*}\StringTok{ }\NormalTok{Estimateur_KM[}\DecValTok{2}\NormalTok{] }\OperatorTok{*}\StringTok{ }\NormalTok{(}\DecValTok{1} \OperatorTok{-}\StringTok{ }\NormalTok{Estimateur_KM[}\DecValTok{2}\NormalTok{])}
\NormalTok{    var_Z <-}\StringTok{ }\NormalTok{n }\OperatorTok{*}\StringTok{ }\NormalTok{(Estimateur_KM[}\DecValTok{2}\NormalTok{] }\OperatorTok{-}\StringTok{ }\NormalTok{Estimateur_KM[}\DecValTok{3}\NormalTok{] }\OperatorTok{*}\StringTok{ }\NormalTok{(}\DecValTok{1} \OperatorTok{-}\StringTok{ }\NormalTok{Estimateur_KM[}\DecValTok{2}\NormalTok{]}
            \OperatorTok{+}\StringTok{ }\NormalTok{Estimateur_KM[}\DecValTok{3}\NormalTok{]))}
\NormalTok{    cov <-}\StringTok{ }\OperatorTok{-}\NormalTok{n }\OperatorTok{*}\StringTok{ }\NormalTok{(}\DecValTok{1} \OperatorTok{-}\StringTok{ }\NormalTok{Estimateur_KM[}\DecValTok{2}\NormalTok{]) }\OperatorTok{*}\StringTok{ }\NormalTok{(Estimateur_KM[}\DecValTok{2}\NormalTok{] }\OperatorTok{-}\StringTok{ }\NormalTok{Estimateur_KM[}\DecValTok{3}\NormalTok{])}
    
\NormalTok{    (var_Y}\OperatorTok{*}\NormalTok{(cj}\OperatorTok{-}\NormalTok{ci)}\OperatorTok{^}\DecValTok{2} \OperatorTok{+}\StringTok{ }\NormalTok{var_Z}\OperatorTok{*}\NormalTok{(x}\OperatorTok{-}\NormalTok{ci)}\OperatorTok{^}\DecValTok{2} \OperatorTok{+}\StringTok{ }\DecValTok{2}\OperatorTok{*}\NormalTok{cov}\OperatorTok{*}\NormalTok{(cj}\OperatorTok{-}\NormalTok{ci)}\OperatorTok{*}\NormalTok{(x}\OperatorTok{-}\NormalTok{ci)) }\OperatorTok{/}\StringTok{ }\NormalTok{(n}\OperatorTok{*}\NormalTok{(cj}\OperatorTok{-}\NormalTok{ci))}\OperatorTok{^}\DecValTok{2}
\NormalTok{\}}

\CommentTok{# Intervalle de confiance au niveau 95% pour S(50)}
\NormalTok{IC_KM <-}\StringTok{ }\KeywordTok{cbind}\NormalTok{(Sn_}\DecValTok{50} \OperatorTok{-}\StringTok{ }\KeywordTok{qnorm}\NormalTok{(}\FloatTok{0.975}\NormalTok{) }\OperatorTok{*}\StringTok{ }\KeywordTok{sqrt}\NormalTok{(Var_Sn_}\DecValTok{50}\NormalTok{),}
\NormalTok{              Sn_}\DecValTok{50} \OperatorTok{+}\StringTok{ }\KeywordTok{qnorm}\NormalTok{(}\FloatTok{0.975}\NormalTok{) }\OperatorTok{*}\StringTok{ }\KeywordTok{sqrt}\NormalTok{(Var_Sn_}\DecValTok{50}\NormalTok{))}
\end{Highlighting}
\end{Shaded}

\subsubsection{\texorpdfstring{c) Intervalle de confiance pour
\(S_n(50)\) avec la transformation log
(-log)}{c) Intervalle de confiance pour S\_n(50) avec la transformation log (-log)}}\label{c-intervalle-de-confiance-pour-s_n50-avec-la-transformation-log--log-1}

\begin{Shaded}
\begin{Highlighting}[]
\CommentTok{# La valeur estimée de la fonction de survie et la variance estimée sont }
\CommentTok{#toujours les mêmes que celles de la section b). Par conséquent, }
\CommentTok{# seulement l'intervalle de confiance est modifié :}
\NormalTok{u <-}\StringTok{ }\KeywordTok{exp}\NormalTok{(}\KeywordTok{qnorm}\NormalTok{(}\FloatTok{0.975}\NormalTok{) }\OperatorTok{*}\StringTok{ }\KeywordTok{sqrt}\NormalTok{(Var_Sn_}\DecValTok{50}\NormalTok{) }\OperatorTok{/}\StringTok{ }\NormalTok{Sn_}\DecValTok{50} \OperatorTok{/}\StringTok{ }\KeywordTok{log}\NormalTok{(Sn_}\DecValTok{50}\NormalTok{))}

\NormalTok{IC_log_KM <-}\StringTok{ }\KeywordTok{cbind}\NormalTok{(Sn_}\DecValTok{50}\OperatorTok{^}\NormalTok{(}\DecValTok{1}\OperatorTok{/}\NormalTok{u),}
\NormalTok{              Sn_}\DecValTok{50}\OperatorTok{^}\NormalTok{u)}
\end{Highlighting}
\end{Shaded}


\end{document}
